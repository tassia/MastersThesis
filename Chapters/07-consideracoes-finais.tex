\chapter{Considera��es finais} \label{chapter:conclusoes}

Resumo geral do documento, principais contribui��es do trabalho, defici�ncias,
e limita��es.

\section{Trabalhos futuros}

\begin{enumerate}
\item An�lise de riscos de privacidade do AppRecommender;
\item Integra��o com AppStream;
\item Considera��o de informa��es temporais para pontua��o multivalorada
\item Compara��o sistem�tica no reposit�rio de pacotes debian da sa�da do
AppRecommender com a lista de pacotes recomendados e sugeridos por cada pacote.
An�lise proposta por Joey Hess em
2008\footnote{\url{http://www.mail-archive.com/debian-devel@lists.debian.org/msg260184.html}}.
\item Testes com o algoritmo OPF para agrupamento \cite{Rocha:09}
\end{enumerate}

%A conclus�o do trabalho ser� pautada pela an�lise dos resultados obtidos com
%a aplica��o do \textit{survey}, seguida por uma proposta de integra��o de um
%recomendador com a infraestrutura de pacotes do projeto Debian, vislumbrando
%uma solu��o multi-distribui��o.
