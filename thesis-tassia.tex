\documentclass{thesis}

\title{AppRecommender: um recomendador \\de aplicativos GNU/Linux}
\author{T�ssia Cam�es Ara�jo}
\email{tassia@gmail.com}
\date{\today}
\location{S�o Paulo}
\blurb{
  Instituto de Matem�tica e Estat�stica \\
  Universidade de S�o Paulo \\[1em]
  Disserta��o apresentada ao Programa de P�s-gradua��o \\
  em Ci�ncia da Computa��o para obten��o do \\
  t�tulo de mestre em ci�ncias \\[1em]
  Orientador: Prof. Dr. Arnaldo Mandel
}

\begin{document}
\maketitle
\begin{abstract}
  A crescente oferta de programas de c�digo aberto na rede mundial de
  computadores exp�e potenciais usu�rios a in�meras possibilidades de escolha.
  Em face da pluralidade de interesses destes indiv�duos, mecanismos eficientes
  que os aproximem daquilo que buscam trazem benef�cios para eles pr�prios,
  assim como para os desenvolvedores dos programas. O \emph{AppRecommender} �
  um recomendador de aplicativos GNU/Linux que realiza uma filtragem no
  conjunto de programas dispon�veis e oferece sugest�es individualizadas para
  os usu�rios. Tal feito � alcan�ado por meio da an�lise de perfis e descoberta
  de padr�es de comportamento na popula��o estudada, de sorte que apenas os
  aplicativos considerados mais suscet�veis a aceita��o sejam oferecidos aos
  usu�rios.
\end{abstract}
\selectlanguage{english}
\begin{abstract}
  The increasing availability of open source software on the World Wide Web
  exposes potential users to a wide range of choices. Given the individuals
  plurality of interests, mechanisms that get them close to what they are
  looking for would benefit themselves and the software developers as well.
  \emph{AppRecommender} is a recommender system for GNU/Linux applications
  which performs a filtering on the set of available software and individually
  offers suggestions to users. This is achieved by analyzing profiles and
  discovering patterns of behavior of the studied population, in a way that
  only those applications considered most prone to acceptance are presented to
  users.
\end{abstract}

\selectlanguage{portuguese}
\pagestyle{plain}
\cleardoublepage
\tableofcontents
\listoffigures
\listoftables

\section{Introdu��o} \label{chapter:introducao}

O universo de programas livres e de c�digo aberto oferece aos usu�rios uma
grande amplitude e diversidade de op��es no que diz respeito a aplicativos
para complementar seus sistemas. No entanto, muitas dessas alternativas
permanecem em relativa obscuridade, pois o car�ter majoritariamente n�o
comercial desses sistemas se reflete na aus�ncia de propaganda e outras formas
de divulga��o ostensiva. Desta forma, a descoberta de programas �teis para um
determinado usu�rio por vezes empaca no excesso de informa��es dispon�veis e
organiza��o inadequada. � costume referir-se a esse fen�meno (p. ex.,
\cite{Iyengar:10}) como ``mais � menos'', no sentido de que o aumento da
disponibilidade de escolhas pode confundir o usu�rio e diminuir sua satisfa��o.

Neste contexto de muitas possibilidades onde poucas s�o de fato atrativas, um
sistema capaz de recomendar aplicativos que presumidamente s�o objeto de
interesse de usu�rios exerceria um papel importante. Desenvolvedores se
beneficiariam por meio de um consequente aumento na utiliza��o de seus
programas que, por serem experimentados por mais usu�rios, certamente
receberiam mais relat�rios de erro (\textit{bug reports}), sugest�es e
contribui��es diversas. Para os usu�rios o benef�cio seria alcan�ado de forma
mais direta, dado que poupariam tempo e recursos outrora dedicados a buscas
e filtragens manuais para encontrar os aplicativos mais adequados a seu
ambiente de trabalho.

Tais benef�cios motivaram a concep��o do \textit{AppRecommender}, um
recomendador de aplicativos GNU/Linux desenvolvido no �mbito de um trabalho de
mestrado, cujo objetivo principal � a experimenta��o de diferentes estrat�gias
para recomenda��o no contexto de componentes de software.

O presente trabalho est� organizado da seguinte forma: os cap�tulos
\ref{chapter:distribuicoes} e \ref{chapter:recomendacao} trazem uma breve
introdu��o sobre distribui��es GNU/Linux e sistemas de recomenda��o. O cap�tulo
\ref{chapter:app_recommender} apresenta o AppRecommender como solu��o em
desenvolvimento para o problema exposto. Em seguida, no cap�tulo
\ref{chapter:trabalhos_correlatos} trabalhos correlatos s�o apresentados e, por
fim, o cap�tulo \ref{chapter:conclusao} traz considera��es finais sobre o
trabalho at� a atual etapa de execu��o.

\input{Chapters/02-recomendacao}
\newpage
\section{Distribui��es GNU/Linux} \label{sec:distribuicoes}

Em 1983 Richard Stallman criou o projeto GNU\footnote{\url{hhtp://www.gnu.org}}
com o objetivo principal de desenvolver um sistema operacional livre em
alternativa ao UNIX\footnote{\url{http://www.unix.org/}} -- solu��o comercial
amplamente difundida na ind�stria -- e que fosse compat�vel com os padr�es
POSIX\footnote{Acr�nimo para \textit{Portable Operating System Interface}, � uma
fam�lia de normas definidas pelo IEEE com foco na portabilidade entre sistemas
operacionais. Dispon�vel em \url{http://standards.ieee.org/develop/wg/POSIX.html}}.
Nos anos 90 o projeto GNU j� havia atra�do muitos colaboradores, que num curto
espa�o de tempo haviam desenvolvido in�meros aplicativos para compor o sistema
operacional. No entanto, o desenvolvimento do n�cleo do sistema (\textit{GNU
Hurd}) n�o acompanhou o ritmo dos demais aplicativos.

Em outubro de 1991 o estudante finland�s Linus Torvalds publicou a vers�o
0.02 do Freax, o n�cleo de um sistema operacional (\textit{kernel}, em ingl�s)
desenvolvido por ele na universidade. Nem o pr�prio Linus imaginava que aquele
projeto, desenvolvido sem grandes pretens�es, teria a dimens�o do que hoje
conhecemos como Linux \cite{Linus:01}.

Com o an�ncio de Torvalds, Stallman vislumbrou a possibilidade de acelerar o
lan�amento do sistema operacional livre, se os aplicativos GNU que j� estavam
prontos fossem combinados com o n�cleo rec�m-lan�ado -- de fato, a primeira
vers�o est�vel do GNU Hurd foi lan�ada apenas em 2001. Em 1992 o Linux foi
licenciado sob a GNU GPL\footnote{Acr�nimo para \textit{General Public
License}, � um suporte legal para a distribui��o livre de softwares.} e as
equipes dos dois projetos come�aram a trabalhar na adapta��o do kernel Linux
para o ambiente GNU. Este esfor�o conjunto desencadeou o surgimento das
distribui��es GNU/Linux, que s�o varia��es do sistema operacional composto por
milhares de aplicativos majoritariamente desenvolvidos pelo projeto GNU e pelo
kernel Linux.

Distribui��es GNU/Linux, como Debian, Fedora, Mandriva e Ubuntu, oferecem
diferentes ``sabores'' do sistema operacional, constitu�dos por aplicativos
selecionados por seus desenvolvedores. As distribui��es reduzem a
complexidade de instala��o e atualiza��o do sistema para usu�rios finais
\cite{Cosmo:08}. Os mantenedores da distribui��o atuam como intermedi�rios
entre os usu�rios e os autores dos softwares (chamados de \textit{upstreams}),
atrav�s do encapsulamento de componentes de software em abstra��es denominadas
\textit{pacotes}.

O processo de desenvolvimento e manuten��o de uma distribui��o varia bastante
de uma para outra e est� diretamente ligado � constitui��o do projeto. Quando
s�o criadas por empresas, costumam receber colabora��o dos usu�rios de
forma limitada, visto que as decis�es-chave s�o tomadas dentro da organiza��o.
Este � o modelo de desenvolvimento descrito por \cite{Raymond:99} como catedral.
Por outro lado, os projetos criados independentemente, formam ao longo do tempo
uma comunidade de desenvolvedores interessados em colaborar, sendo estes os
�nicos respons�veis pelo sucesso ou fracasso do projeto. Neste modelo,
denominado bazar, o c�digo-fonte est� dispon�vel durante todo o processo
de desenvolvimento, n�o apenas nos lan�amentos, permitindo que a contribui��o
seja mais efetiva. Nesses casos observa-se com mais clareza o fen�meno do
consumidor produtor descrito anteriormente. Segundo o autor, este modelo �
mais favor�vel ao sucesso, pois um bom trabalho de desenvolvimento de software
� motivado por uma necessidade pessoal do desenvolvedor (ou seja, o
desenvolvedor � tamb�m usu�rio).

A sele��o e configura��o dos aplicativos b�sicos de uma distribui��o ficam
sob a responsabilidade da equipe que a desenvolve, com diferentes n�veis de
interfer�ncia da comunidade, como descrito acima. Este � um ponto crucial do
desenvolvimento, visto que � um dos fatores que mais influenciam a escolha dos
usu�rios pela distribui��o. O impacto que tal sele��o tem para a comunidade de
usu�rios resulta em frequentes pol�micas em torno do tema, como a gerada pelo
an�ncio de que o projeto \textit{Ubuntu} abandonaria o \textit{Gnome} como
interface padr�o de usu�rio \cite{Paul:10}.

Softwares adicionais para atender a demandas espec�ficas dos usu�rios
devem ser instalados pelos mesmos, ap�s a configura��o do sistema operacional.
A infraestrutura de instala��o de softwares provida pela distribui��o,
geralmente baseada em pacotes, facilita este processo \cite{Cosmo:08}. Ainda
assim, a sele��o dos programas � de responsabilidade do usu�rio do sistema.

\subsection{Escolha da plataforma}

A distribui��o escolhida como base para o desenvolvimento deste trabalho
foi o Debian GNU/Linux. No entanto, a codifica��o ser� realizada com o maior
grau poss�vel de independ�ncia de plataforma, com o intuito de que os
resultados sejam facilmente adapt�veis para outros contextos. As seguir est�o
descritos os crit�rios que pautaram esta escolha.

\begin{enumerate}

\item \textbf{Esquema consistente de distribui��o de aplicativos.} O
  gerenciamento de pacotes em sistemas Debian GNU/Linux � realizado atrav�s do
  \textit{APT (Advanced Packaging
  Tool)}\footnote{\url{http://wiki.debian.org/Apt}}. A��es como a busca,
  obten��o, instala��o, atualiza��o e remo��o de pacotes s�o disparadas pelo
  \textit{APT}, que num n�vel mais baixo faz uso do \textit{dpkg}, ferramenta
  que de fato realiza instala��es e remo��es de softwares. O \textit{APT}
  tamb�m gerencia de maneira eficiente as rela��es de conflito e depend�ncia
  entre pacotes. Ao receber um pedido de modifica��o da configura��o do sistema
  -- por exemplo, instala��o de um novo componente -- o \textit{APT} tenta
  satisfazer a requisi��o a partir do conhecimento de como obter os componentes
  (endere�o dos reposit�rios de pacotes) e das rela��es de depend�ncia entre os
  mesmos. Desta forma, o gerenciador promove a instala��o de todas as
  depend�ncias de um pacote antes de instal�-lo, ao passo que n�o permite a
  instala��o de pacotes que conflitem com outros j� instalados no sistema.

\vspace{0.3cm}
\item \textbf{Disponibilidade de dados estat�sticos.} O \textit{Popcon
  (Popularity Contest)}\footnote{\url{http://popcon.debian.org}} � a concurso
  de popularidade entre pacotes. Os usu�rios que aceitam participar do concurso
  enviam periodicamente a sua lista de pacotes instalados no sistema, que s�o
  armazenados no servidor do Popcon. Diariamente as listas recebidas s�o
  processadas e dados estat�sticos acerca do uso dos pacotes s�o gerados e
  disponibilizados no website do projeto.

\vspace{0.3cm}
\item \textbf{Possibilidade de integra��o dos resultados do trabalho.} Segundo
  o \textit{contrato social
  Debian}\footnote{\url{http://www.debian.org/social_contract.pt.html}},
  o desenvolvimento do projeto � guiado pelas necessidades dos usu�rios e da
  comunidade. Portanto, as iniciativas de colaboradores individuais, sejam eles
  desenvolvedores oficiais ou n�o, ser�o igualmente consideradas e passar�o a
  fazer parte da distribui��o se seguirem os princ�pios do projeto e forem
  considerados �teis para a comunidade.

\vspace{0.3cm}
\item \textbf{Popularidade.} O Debian � um projeto de destaque no ecossistema
  do software livre. Desde o lan�amento da primeira vers�o de sua distribui��o,
  em 1993, o projeto cresceu bastante em termos de componentes de software
  (atualmente prov� mais de 25.000 pacotes), colaboradores e usu�rios. A
  \textit{Distrowatch}, que tem 323 distribui��es ativas em sua base de
  dados\footnote{Consulta realizada em 24 de janeiro de 2011.}, classifica o
  Debian GNU/Linux entre as 10 distribui��es mais
  populares\footnote{\url{http://distrowatch.com/dwres.php?resource=major}}.
  O Debian aparece na quinta posi��o em suas estat�sticas de p�ginas
  visitadas\footnote{\url{http://distrowatch.com/stats.php?section=popularity}}.
  J� o \textit{Linux Counter\footnote{\url{http://counter.li.org/reports/machines.php}}} apresenta o Debian como a segunda distribui��o mais popular entre as
  m�quinas cadastradas que rodam o kernel Linux ($16\%$), ficando atr�s apenas
  do Ubuntu\footnote{\url{http://www.ubuntu.com/community/ubuntu-and-debian}}
  ($24\%$), que � uma distribui��o derivada do Debian. Nas pesquisas da
  \textit{W$^{\textrm{3}}$Techs} sobre tecnologias para servi�os
  web\footnote{\url{http://w3techs.com/technologies/history_details/os-linux}},
  o Debian aparece em segundo lugar, estando presente em $27\%$ dos servidores.
  Na primeira posi��o est� o CentOS com $31\%$.

\end{enumerate}

De maneira geral, quando o projeto Debian � mencionado trata-se n�o somente do
sistema operacional, mas de toda a infra-estrutura de desenvolvimento e
coordena��o que d� suporte ao trabalho de cerca de 900 desenvolvedores
oficiais\footnote{\url{http://www.perrier.eu.org/weblog/2010/08/07\#devel-countries-2010}},
al�m de outros milhares de colaboradores ao redor do globo. O trabalho �
realizado de forma colaborativa, afinado pelo objetivo comum de produzir e
disponibilizar livremente um sistema operacional de qualidade para seus
usu�rios \cite{Jackson:98}. A intera��o entre os desenvolvedores acontece
majoritariamente atrav�s da Internet, por meio de canais IRC e listas de
discuss�o p�blicas. N�o existe uma entidade formal ou qualquer tipo de
organiza��o que concentre, coordene ou defina as atividades do projeto. O que
observa-se � um modelo de governan�a consolidado que emergiu naturalmente ao
longo de sua hist�ria \cite{Ferraro:07}.

\section{Recomenda��o nas distribui��es} \label{sec:rec_distro}

Grande parte das distribui��es GNU/Linux t�m investido no desenvolvimento de
interfaces para facilitar o gerenciamento de aplicativos e a forma como se obt�m
informa��es sobre os mesmos. Entre os dias 18 e 21 de janeiro 2011 aconteceu a
primeira reuni�o sobre a tem�tica com a presen�a de desenvolvedores de
distribui��es variadas (\textit{Cross-distribution Meeting on Application
Installer}). O encontro teve como principais objetivos a defini��o de padr�es
entre os diferentes projetos no que diz respeito a: procedimentos de instala��o
de aplica��es; metadados associados aos pacotes; o modo como tais informa��es
devem ser geradas e armazenadas; protocolo para manuten��o de metadados
din�micos; e a defini��o de quais metadados devem ser compartilhados entre as
distribui��es, em detrimento de outros considerados espec�ficos de cada projeto
\cite{Freedesktop:11}.

O projeto Debian tem se destacado no universo das distribui��es por suas
iniciativas pioneiras no campo de gerenciamento de aplica��es \cite{Zini:11}.
Diante da complexa e crescente estrutura do projeto, observa-se um esfor�o por
parte dos desenvolvedores, principalmente da equipe respons�vel pelo controle
de qualidade\footnote{\url{http://qa.debian.org}}, de reunir, organizar e
disponibilizar as informa��es ou meta-dados concernentes a esta estrutura
\cite{Nussbaum:10}.

A tabela \ref{tab:recDebian} relaciona algumas destas iniciativas que est�o
diretamente ligadas ao gerenciamento de pacotes. Vale ressaltar que a maioria
destas solu��es foi inicialmente desenvolvida num contexto extra-oficial e
ao passo que se mostraram �teis e eficientes foram absorvidas pela comunidade
de usu�rios e desenvolvedores.

\begin{sidewaystable}
  \centering
  \caption{Recomendadores no Debian}
  \label{tab:recDebian}
  \small
  \newcommand\T{\rule{0pt}{3.0ex}}
  \newcommand\B{\rule[-1.8ex]{0pt}{0pt}}
  \begin{tabularx}{25cm}{| c | X | c |}
    \hline
    \normalsize \textbf{\textit{Solu��o}}\T\B & \normalsize \textbf{\textit{Descri��o}} & \normalsize \textbf{\textit{Estrat�gia de recomenda��o}}\\
    \hline BTS \T & \multirow{3}{13cm}{Sistema de acompanhamento de \textit{bugs}, alimentado pelos usu�rios e desenvolvedores da distribui��o. Re�ne todo o hist�rico referente ao relat�rio e corre��o de erros em pacotes.} & \multirow{10}{*}{Reputa��o}\\
    \textit{Bug Tracking System}& & \\
    \scriptsize{\url{http://bugs.debian.org}} & & \\
    & & \\
    \cline{1-2}
    Popcon \T & \multirow{3}{13cm}{Concurso de popularidade entre pacotes realizado diariamente, disponibiliza estat�sticas de uso dos pacotes do reposit�rio. Prov� gr�ficos espec�ficos por pacote, por arquitetura, por desenvolvedor etc.} & \\
    \textit{Popularity Contest}& & \\
    \scriptsize{\url{http://popcon.debian.org}} & & \\
    & & \\
    \cline{1-2}
    PTS \T & \multirow{3}{13cm}{Sistema de acompanhamento de pacotes, re�ne informa��es relativas � manuten��o dos pacotes: vers�o, mantenedor, \textit{upstream}, \textit{bugs} abertos por tipo, �ltimas atualiza��es no reposit�rio etc.} & \\
    \textit{Package Tracking System}& & \\
    \scriptsize{\url{http://packages.qa.debian.org}} & & \\
    & & \\
    \cline{1-2}
    UDD \T & \multirow{3}{13cm}{{Iniciativa recente do time de qualidade criada com o intuito de reunir informa��es de diversos aspectos do Debian numa base de dados �nica. Usu�rios avan�ados podem consultar esta base para tomar decis�es acerca de que pacotes utilizar.}} & \\
    \textit{Ultimate Debian Database}& & \\
    \scriptsize{\url{http://udd.debian.org}}& & \\
    & & \\
    \hline
    Debtags \T & \multirow{2}{13.5cm}{Caracteriza��o dos pacotes por m�ltiplos atributos, realizada manualmente por usu�rios e desenvolvedores, que auxilia a navega��o e busca no reposit�rio de pacotes.} & \multirow{2}{*}{Baseada em conte�do}\\
    Classifica��o de pacotes & & \\
    \cline{2-3}
    \scriptsize{\url{http://debtags.alioth.debian.org}} & \multirow{2}{*}{Novas \textit{tags} (atributos) s�o sugeridas a partir das \textit{tags} j� associadas ao pacote.} & \multirow{2}{*}{Associa��o}\\
    & & \\
    \hline
  \end{tabularx}
\end{sidewaystable}

Dois esfor�os anteriores de desenvolvimento de recomendadores de pacotes Debian
foram identificados, por�m ambos descontinuados. A primeira foi o
\textit{PopSuggest}\footnote{http://www.enricozini.org/2007/debtags/popcon-play/},
que oferecia recomenda��es a partir de dados do \textit{Popcon} como uma
ilustra��o das possibilidades de uso dos dados coletados. A outra foi o
\textit{Debommender}\footnote{http://ostatic.com/debommender}, desenvolvido
como prova de conceito no �mbito de um trabalho de gradua��o, n�o sendo por�m
integrado aos servi�os da distribui��o.

%FIXME [Descrever o software-center e outras solucoes]
% http://distributions.freedesktop.org/wiki/Meetings/AppInstaller2011
%Software Center: UI focused on applications
%app-install: UI focused on applications + code to generate metadata
%screenshots.debian.net: web service providing app screenshot
%Open Collaboration Services: specification that can help provide social features (rating, comments, etc.)
%http://launchpad.net/rnr-server: django based server for ratings&reviews webservice (not following the open-collaboration-spec yet, but there is interesst about this from the devs)
%mageia-app-db: Mageia's application database (work in progress, there's a link to a nightly updated instance on the wiki)
%Sophie: Search and analyze rpms from various distribution.
%openSUSE Software Portal: a web app focused on applications

%[Pacotes Debian, Rela��o entre pacotes, O Reposit�rio de Pacotes]

%\subsection{Pacotes Debian}
%. Bin�rios e fontes\\
%. upstream, maintainer, uploader\\
%. Prioridade: required, important, standard, optional, extra\\
%. Base system = required or important (muitos s�o marcados como essenciais)\\
%. Essenciais: o gerenciador de pacotes se recusa a remover, a menos que seja for�ado\\
%
%\subsection{Rela��o entre pacotes}
%. Depend�ncia: Pre-depends, Depends, Recommends, Suggests, Enhances\\
%. Anti-depend�ncia: Breaks, Conflicts, Replaces\\
%. Pacotes virtuais: Provides\\
%. Entre fontes e bin�rios: Build-Depends, Build-Depends-Indep, Build-Conflicts, Build-Conflicts-Indep
%
%\subsection{O Reposit�rio de Pacotes}
%. The Debian Archive,\\
%. �reas: main, contrib, non-free\\
%. Se��es\\


\input{Chapters/04-metodologia}
\input{Chapters/05-conclusao}

\bibliographystyle{sbc}
\bibliography{thesis-tassia}

\end{document}
